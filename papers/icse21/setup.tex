\section{Experiment Setup}
\label{sec:settings}

In this section, we describe our experiment setup for comparing the
\methodologies.  This setup is generic and could be used for any task.
%
In this work we instantiate it for \comgen and \methnam tasks.  
\Fix{In the intro you said you do two code summarization tasks.}
The
goal of the experiment is to train a model following each \methodology
and evaluate and compare the models on a \emph{test set that is
  common} to all \methodologies.  We record automatic metrics and
reason about their differences.  \Fix{no prior work used human eval}

We define notations, by extending Section~\ref{sec:prelim}, to help us
accurately describe our experiment setup.  Let $\mathcal{P}$ be the
set of projects we intend to collect data from.  We then perform the
split $\mathcal{P}_{train}, \mathcal{P}_{val}, \mathcal{P}_{test} =
\asplit(\mathcal{P}, 80\%, 10\%, 10\%)$.  The ratio of 80\%/10\%/10\%
is chosen following machine learning common
practice~\cite{StanfordSplitting}.

Let $\mathcal{D}(\mathcal{P}, \atime) = \sum_{P\in\mathcal{P}}
\aextract(\atime, P)$ be the data collected from projects in
$\mathcal{P}$ at time point $t$.  The exact implementation of
$\mathcal{D}(\cdot, \cdot)$ is task-specific.

Let $\mathcal{D}(\mathcal{P}, \atimep, \atime)$ be the new data from
projects in $\mathcal{P}$ after time point \atimep{} before time point
\atime{}.  To compute $\mathcal{D}(Set[Project], Time, Time)$, let
$F(d_1, d_2)$ be a \filterfunc that returns a subset of $d_2$ after
removing the data exists in $d_1$.  The exact implementation of
$F(Set, Set)$ is task-specific.  Then, we have
$\mathcal{D}(\mathcal{P}, \atimep, \atime) =
F(\mathcal{D}(\mathcal{P}, \atimep), \mathcal{D}(\mathcal{P},
\atime))$.

Given $\mathcal{D}(\mathcal{P}, \atime)$ and $\mathcal{D}(\mathcal{P},
\atimep, \atime)$, we perform the splits:
\begin{flalign*}
  &\mathcal{D}_{train}(\mathcal{P}, \atime), \mathcal{D}_{val}(\mathcal{P}, \atime), \mathcal{D}_{test}(\mathcal{P}, \atime) \\
  &\quad = \asplit(\mathcal{D}(\mathcal{P}, \atime), 80\%, 10\%, 10\%) \\
%
  &\mathcal{D}_{train}(\mathcal{P}, \atimep, \atime), \mathcal{D}_{val}(\mathcal{P}, \atimep, \atime), \mathcal{D}_{test}(\mathcal{P}, \atimep, \atime) \\
  &\quad = \asplit(\mathcal{D}(\mathcal{P}, \atimep, \atime), 80\%, 10\%, 10\%)
\end{flalign*}
of the data respectively, and ensures that
$$
\forall x \in \{train, val, test\}, \mathcal{D}_x(\mathcal{P}, \atimep, \atime) \subseteq \mathcal{D}_x(\mathcal{P}, \atime)
$$

\begin{figure*}[t]
  \centering
  \begin{tikzpicture}[
  line width=0.4pt,
  node distance=0ex and 0em,
  every node/.style={scale=0.9},
  gridBox/.style={rectangle, opacity=0, draw=red},
  box/.style={rectangle, draw=black, inner sep=2pt, font=\small},
  rounded box/.style={rectangle, rounded corners, draw=black, fill=gray!20, inner sep=3pt, font=\small},
  anno/.style={font=\footnotesize},
]

  % legends

  \draw[dsTrain] (-4em,-2ex) rectangle (-3em,-3ex);
  \node (b-l-train) at (-4.2em,-2.5ex) [anchor=east, anno] {\Train};
  \draw[dsVal] (-4em,-4ex) rectangle (-3em,-5ex);
  \node (b-l-val) at (-4.2em,-4.5ex) [anchor=east, anno] {\Val};
  \draw[dsTest] (-4em,-6ex) rectangle (-3em,-7ex);
  \node (b-l-test) at (-4.2em,-6.5ex) [anchor=east, anno] {\Test};
  

  \foreach \s/\xbase in {mixedproj/0,crossproj/14em,evoaware/28em} {
    
    \node[coordinate] (\s-p1) at (\xbase,0) [] {};
    \node[coordinate] (\s-p2) [right = 6em of \s-p1] [] {};
    \node[coordinate] (\s-p3) [right = 8em of \s-p1] [] {};
    \node[coordinate] (\s-p4) [right = 10em of \s-p1] [] {};

    \node[coordinate] (\s-t1) at (\xbase,0) [] {};
    \node[coordinate] (\s-t2) [below = 6ex of \s-t1] [] {};
    \node[coordinate] (\s-t3) [below = 12ex of \s-t1] [] {};
    \node[coordinate] (\s-t4) [below = 18ex of \s-t1] [] {};

    \foreach \x/\y in {1/2,2/3,3/4} {
      \node[coordinate] (\s-t\x-1) at ($(\s-t\x)!0.5!(\s-t\y)$) [] {};
      \node[coordinate] (\s-t\x-2) at ($(\s-t\x)!0.75!(\s-t\y)$) [] {};
    };

    \foreach \x/\t in {1/2017,2/2018,3/2019,4/2020} {
      \node (\s-b-t\x) [left = .2em of \s-t\x] [anno] {\t};
    };

    \draw[line width=1pt] (\s-p4) rectangle (\s-t4);

    \foreach \x in {2,3} {
      \draw (\s-t1 -| \s-p\x) -- (\s-t4 -| \s-p\x);
    };

    \foreach \x in {2,3} {
      \draw (\s-t\x -| \s-p1) -- (\s-t\x -| \s-p4);
    };

    \foreach \x in {1-1,1-2,2-1,2-2,3-1,3-2} {
      \draw[dashed] (\s-t\x -| \s-p1) -- (\s-t\x -| \s-p4);
    }

    \node[coordinate] (\s-p1-base) at ($(\s-p1|-\s-t4)!0.5!(\s-p2|-\s-t4)$) [] {};
    \node[coordinate] (\s-p2-base) at ($(\s-p2|-\s-t4)!0.5!(\s-p3|-\s-t4)$) [] {};
    \node[coordinate] (\s-p3-base) at ($(\s-p3|-\s-t4)!0.5!(\s-p4|-\s-t4)$) [] {};
    \foreach \x/\ds in {1/train,2/val,3/test} {
      \node (\s-b-p\x) [below = .5ex of \s-p\x-base] [anno] {$\mathcal{P}_{\ds}$};
    };

    \node[coordinate] (\s-south) at ($(\s-p1|-\s-t4)!0.5!(\s-p4|-\s-t4)$) [] {};
  };

  \node (mixedproj-b-text) [below = 4ex of mixedproj-south] [rounded box] {\Mixedproj};
  \node (crossproj-b-text) [below = 4ex of crossproj-south] [rounded box] {\Crossproj};
  \node (evoaware-b-text) [below = 4ex of evoaware-south] [rounded box] {\Evoaware};

  \foreach \x in {1,2,3} {
    \draw[dsTrain, draw=none] (mixedproj-p1|-mixedproj-t\x) rectangle (mixedproj-p4|-mixedproj-t\x-1);
    \draw[dsVal, draw=none] (mixedproj-p1|-mixedproj-t\x-1) rectangle (mixedproj-p4|-mixedproj-t\x-2);
  }
  \draw[dsTest, draw=none] (mixedproj-p4|-mixedproj-t3-2) rectangle (mixedproj-p3|-mixedproj-t4);

  \draw[dsTrain, draw=none] (crossproj-p2) rectangle (crossproj-p1|-crossproj-t4);
  \draw[dsVal, draw=none] (crossproj-p3) rectangle (crossproj-p2|-crossproj-t4);
  \draw[dsTest, draw=none] (crossproj-p4|-crossproj-t3-2) rectangle (crossproj-p3|-crossproj-t4);

  \draw[dsTrain, draw=none] (evoaware-p1|-evoaware-t1) rectangle (evoaware-p4|-evoaware-t2);
  \draw[dsVal, draw=none] (evoaware-p1|-evoaware-t2) rectangle (evoaware-p4|-evoaware-t3);
  \draw[dsTest, draw=none] (evoaware-p4|-evoaware-t3-2) rectangle (evoaware-p3|-evoaware-t4);

\end{tikzpicture}

  \caption{Evaluation settings. \JiyangComment{It is a bit confusing
      to include time for Mixed-project, I suggest removing time in y
      axis.}\label{fig:eval-settings}}
\end{figure*}

\InputWithSpace{tables/table-eval-settings}

For each \methodology, we design an evaluation setting that specifies
the examples used for \train, \val, and \test.
Figure~\ref{fig:eval-settings} illustrates the \train, \val, and \test
sets for each \methodology, and Table~\ref{table:eval-settings} lists
those sets using the notations we just defined.  $\atimeppp, \atimepp,
\atimep, \atime$ are the chronologically ordered time points at which
we collect the \dataset{s}; in our experiment, we choose $\atimeppp =
\ajanone, 2017; \atimepp = \ajanone, 2018; \atimep = \ajanone, 2019;
\atime = \ajanone, 2020$.

For the \mixedproj \methodology, we use all examples available at
$\atime$ and split them into \train, \val, and \test sets, and
examples in different sets are allowed to come from the same project.
For the \crossproj \methodology, we still use all examples available
at $\atime$ but ensure that examples of one project can be only in one
of \train, \val, or \test set.  For the \evoaware \methodology, we use
the examples between $\atimeppp$ and $\atimepp$ for \train, between
$\atimepp$ and $\atimep$ for \val, and between $\atimep$ and $\atime$
for \test. Specially, the \test set is constrained in a way that only
contains examples that fit all the three \methodologies.

%% The \test sets of the three settings are by natural different, which
%% makes it hard to compare different evaluation settings using the same
%% model and the same metric.  To compare the three evaluation settings
%% in a controlled way, we take the union of their \test sets as the
%% \textit{\common} \test set (denote as $T_{\common}$).  We will refer
%% to the \test set that are specific to each evaluation setting as the
%% \textit{\standard} \test set (denote as $T_{\standard}$).
%% Table~\ref{table:eval-settings} lists the \train, \val, and \standard
%% \test sets for each evaluation setting, as well as the \common \test
%% set.  For simplicity, we use the 4-digits year to represent the time
%% point at the beginning of that year, e.g., 2020 represents Jan 1, 2020
%% 00:00.

%% # Use cases

%% ## Evolution
%% - Train on Data(P_train, 2017, 2018), validation on Data(P_val, 2018, 2019);
%% - Test on Data_test(P_test, 2019, 2020);

%% ## Cross-project
%% - Train on Data(P_train, 2020), validation on Data(P_val, 2020);
%% - Test on Data_test(P_test, 2019, 2020);
%% - Optionally test on Data(P_test, 2020) and show the results are not statistically significantly different than testing on the previous test set;

%% ## Mixed-project
%% - Train on Data_train(P, 2020), validation on Data_val(P, 2020);
%% - Test on Data_test(P_test, 2019, 2020);
%% - Optionally test on Data_test(P, 2020) and show the results are not statistically significantly different than testing on the previous test set;
