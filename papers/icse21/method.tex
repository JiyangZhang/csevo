\section{Methodologies}
\label{sec:methodologies}

% Manually written

\begin{table*}[t]
  \centering
  \caption{\TCPriorWork}
  \begin{tabular}{l|llll ccc}
    \toprule
    & & & & & \multicolumn{3}{c}{\textbf{\Methodology}} \\\cline{6-8}
    \multirow{-2}{*}{\textbf{Task}} &
    \multirow{-2}{*}{\textbf{Authors}} &
    \multirow{-2}{*}{\textbf{Reference}} &
    \multirow{-2}{*}{\textbf{Year}} & \multirow{-2}{*}{\textbf{Language}}&
    \textbf{\Mixedproj}  & \textbf{\Crossproj} & \textbf{\Evoaware} \\
    \midrule
     \multirow{7}{*}{\rotatebox[origin=c]{270}{\makecell[c]{Comment\\Generation}}} &
    Iyer et al. & \cite{IyerETAL16Summarizing} & 2016 & SQL & \mycheckmark & \mycross & \mycross \\
    & Wan et al. & \cite{WanETAL18Improving} & 2018 & Python & \mycheckmark & \mycross & \mycross \\
    & Hu et al. & \cite{HuETAL18Deep,HuETAL19Deep} & 2019 & Java & \mycheckmark & \mycross & \mycross \\
    & LeClair et al. & \cite{LeClairETAL19Neural} & 2019 & Java & \mycross & \mycheckmark & \mycross \\
    & LeClair et al. & \cite{LeClairETAL20Improved} & 2019 & Java& \mycross & \mycheckmark & \mycross \\
    & Ahmad et al. & \cite{AhmadETAL20Transformer-based} & 2019 & Java, Python &\mycheckmark & \mycross & \mycross \\
    & Cai et al. & \cite{CaiETAL20TAG} & 2020 &  SQL, Python &\mycheckmark & \mycross & \mycross \\
    \hline
    \multirow{5}{*}{\rotatebox[origin=c]{270}{\makecell[c]{Method\\Naming}}} &
    Allamanis et al. & \cite{AllamanisETAL16Convolutional} & 2016 & Java & \mycheckmark & \mycross & \mycross \\
    & Xu et al. & \cite{XuETAL19Method} & 2019 & Java & \mycheckmark & \mycross & \mycross \\
    & Yonai et al. & \cite{YonaiETAL19Mercem} & 2019 & Java & \mycheckmark & \mycross & \mycross\\
    & Alon et al. & \cite{AlonETAL19Code2vec} & 2019 & Java & \mycheckmark & \mycross & \mycross \\
    & Alon et al. & \cite{AlonETAL19code2seq} & 2019 & Java & \mycross & \mycheckmark & \mycross\\
    & Nguyen et al. & \cite{NguyenETAL20Suggesting} & 2020 & Java & \mycheckmark & \mycheckmark & \mycross \\
    \bottomrule
  \end{tabular}
  \vspace{\TVPriorWork}
\end{table*}



Table~\ref{table:prior-work} lists recent work on \comgen and
\methnam\Fix{short description}.  Each row shows information for one
paper; two references in a single row indicate that the conference
paper was extended into a journal paper.  The first group of papers
targets \comgen and the second group targets \methnam.  The last three
columns show the methodology (or methodologies) that were used in the
evaluation.
\Fix{I'm confused: none of these methods are code summarization?, which
is the task you said you will look into in the intro.}

In this section, we first summarize two commonly used methodologies:
\mixedproj and \crossproj.  Then, we introduce a novel methodology,
inspired by our development needs and prior work on defect
prediction~\cite{TanETAL15Online}, that uses \evoaware evaluation
dataset.

\subsection{\Mixedproj}

\begin{figure}[t]
  \centering
  \input{figs/method-mixedproj}
  \caption{\Mixedproj \methodology. \label{fig:method-mixedproj}}
\end{figure}

The \emph{\mixedproj} methodology, which is the most commonly used in
prior work based on Table~\ref{table:prior-work}, extracts \examples
at a single point in time from various projects, then randomly
shuffles the \examples and splits them into training, validation, and
testing sets.  More formally, we define the following sets:\Fix{What are x,  y, z?}
%
\begin{align*}
  \aexamples = \bigcup_{\aproject \in \aprojects} \aextract(\atime, \aproject) \\
  \atrain, \aval, \atest = \asplit(\ashuffle(\aexamples), x, y, z)
\end{align*}
%
Figure~\ref{fig:method-mixedproj} illustrates the \mixedproj
methodology\Fix{gray}.
%
This methodology is \emph{time-unaware}, i.e., it does not consider if
\examples in the testing sets are committed into a project before or
after \examples that are in the training or validation sets.
\Fix{What do boxes mean in Figure 1?}

\Fix{talk what sets/\examples might be for different tasks}

\subsection{\Crossproj}

\begin{figure}[t]
  \centering
  \begin{tikzpicture}[
  line width=0.4pt,
  node distance=0ex and 0em,
  every node/.style={scale=0.9},
  gridBox/.style={rectangle, opacity=0, draw=red},
  box/.style={rectangle, draw=black, inner sep=2pt, font=\small},
  rounded box/.style={rectangle, rounded corners, draw=black, inner sep=2pt, font=\small},
  anno/.style={font=\footnotesize},
]

  % legends
  \node (b-l-train) at (0, -2ex) [sExample, dsTrain] {};
  \node (b-l-train-text) [right = .1em of b-l-train] [anno] {\Train};
  \node (b-l-val) [right = .2em of b-l-train-text] [sExample, dsVal] {};
  \node (b-l-val-text) [right = .1em of b-l-val] [anno] {\Val};
  \node (b-l-test) [right = .2em of b-l-val-text] [sExample, dsTest] {};
  \node (b-l-test-text) [right = .1em of b-l-test] [anno] {\Test};

  \node (b-p1) at (0,0) [anchor=south, sProject] {};
  \node (b-p2) [right = \wSepProject of b-p1.east] [sProject] {};
  \node (b-p3) [right = \wSepProject of b-p2.east] [sProject] {};
  \node (b-p4) [right = \wSepProject of b-p3.east] [sProject] {};
  \node (b-p5) [right = \wSepProject of b-p4.east] [sProject] {};

  \foreach \pi/\ds in {1/dsTrain,2/dsTrain,3/dsTrain,4/dsVal,5/dsTest} {
    \foreach \ex/\ey [count=\ei] in {
      -.8em/-1em, 0em/-1em, .8em/-1em,
      -1.2em/0, -.4em/0, .4em/0, 1.2em/0,
      -.8em/1em, 0em/1em, .8em/1em
    } {
      \node (b-p\pi-\ei) [below right = \ey and \ex of b-p\pi.center, anchor=center] [sExample, \ds] {};
    }
  };

\end{tikzpicture}

  \caption{\Crossproj \methodology. \label{fig:method-crossproj}}
\end{figure}

The \emph{\crossproj} methodology extracts examples at a single point
in time from various projects.  This methodology was also commonly
used in prior work.  Unlike the \mixedproj \methodology, \crossproj
splits the \emph{set of projects} into testing, validation, and
training sets.  Thus, all \examples from any project in the
training/validation/testing set will be used for
training/validation/testing.\Fix{This sounds like mixed project. Shouldn't it be the set of projects in test/val is disjoint from those in train?}
%
Figure~\ref{fig:method-crossproj} illustrates this
methodology\Fix{gray}.  More formally, we define the following sets:
%
\begin{align*}
  \aprojecttrain, \aprojectval, \aprojecttest = \asplit(\ashuffle(P), x, y, z) \\
  \atrain = \bigcup_{\aproject \in \aprojecttrain} \aextract(\atime, \aproject) \\
  \aval = \bigcup_{\aproject \in \aprojectval} \aextract(\atime, \aproject) \\
  \atest = \bigcup_{\aproject \in \aprojecttest} \aextract(\atime, \aproject)
\end{align*}
%
The \crossproj methodology is explicitly testing the ability to
transfer a model to new projects.
%
However, \crossproj is also time-unaware, i.e., it does not consider
if \examples from a project that is in testing set come before or
after \examples from projects in the training set.

% I would be surprised if it made a difference since this would imply
% causal contamination across projects which is possible but I assume
% unlikely.

\subsection{\Evoaware}

\begin{figure}[t]
  \centering
  \begin{tikzpicture}[
  line width=0.4pt,
  node distance=0ex and 0em,
  every node/.style={scale=0.9},
  gridBox/.style={rectangle, opacity=0, draw=red},
  box/.style={rectangle, draw=black, inner sep=2pt, font=\small},
  rounded box/.style={rectangle, rounded corners, draw=black, inner sep=2pt, font=\small},
  anno/.style={font=\footnotesize},
]

  % legends
  \node (b-l-train) at (0, -2ex) [sExample, dsTrain] {};
  \node (b-l-train-text) [right = .1em of b-l-train] [anno] {\Train};
  \node (b-l-val) [right = .2em of b-l-train-text] [sExample, dsVal] {};
  \node (b-l-val-text) [right = .1em of b-l-val] [anno] {\Val};
  \node (b-l-test) [right = .2em of b-l-val-text] [sExample, dsTest] {};
  \node (b-l-test-text) [right = .1em of b-l-test] [anno] {\Test};

  \tikzset{sEvoProject/.style={sProject, minimum height=8em, minimum width=3em}}

  \node (b-p1) at (0,0) [anchor=south,sEvoProject] {};
  \node (b-p2) [right = \wSepProject of b-p1.east] [sEvoProject] {};
  \node (b-p3) [right = \wSepProject of b-p2.east] [sEvoProject] {};
  \node (b-p4) [right = \wSepProject of b-p3.east] [sEvoProject] {};
  \node (b-p5) [right = \wSepProject of b-p4.east] [sEvoProject] {};

  \node[coordinate] (c-time-beg) [below left = 1ex and 1em of b-p1.north west] {};
  \node[coordinate] (c-time-end) [above left = 0ex and 1em of b-p1.south west] {};
  \draw[->] (c-time-beg) -- (c-time-end);
  \node (b-time) [above left = -.5ex and .3em of c-time-end] [anno] {time};

  \foreach \pi in {1,2,3,4,5} {
    \foreach \ex/\ey/\ds [count=\ei] in {
      -.8em/-2em/dsTrain, 0em/-2em/dsTrain, .8em/-2em/dsTrain,
      -.8em/0/dsVal,  0em/0/dsVal, .8em/0/dsVal,
      -.8em/2em/dsTest, 0em/2em/dsTest, .8em/2em/dsTest} {
      \node (b-p\pi-\ei) [below right = \ey and \ex of b-p\pi.center, anchor=center] [sExample, \ds] {};
    }
  };

  \node (b-t1) at (c-time-beg |- b-p1-1) [circle, minimum size=.2em, inner sep=0, draw=black, fill=black] {};
  \node (b-t1-text) [left = .2em of b-t1] [anno] {$\tau_1$};
  \node (b-t2) at (c-time-beg |- b-p1-4) [circle, minimum size=.2em, inner sep=0, draw=black, fill=black] {};
  \node (b-t2-text) [left = .2em of b-t2] [anno] {$\tau_2$};
  \node (b-t3) at (c-time-beg |- b-p1-7) [circle, minimum size=.2em, inner sep=0, draw=black, fill=black] {};
  \node (b-t3-text) [left = .2em of b-t3] [anno] {$\tau_3$};

\end{tikzpicture}

  \caption{\Evoaware \methodology. \label{fig:method-evoaware}}
\end{figure}

We propose a novel \methodology, dubbed \emph{\evoaware}.  Unlike the
\methodologies explained earlier in this section, the \evoaware
\methodology is \emph{time-aware}, i.e., \examples in the training set
were available in software repositories \emph{before} \examples in the
validation set, which were in turn available \emph{before} examples in
the testing set.  Figure~\ref{fig:method-evoaware} illustrates this
\methodology.  This \methodology could be used in combination with
\mixedproj or \crossproj.  More formally, we define the following
steps: \JiyangComment{Do we want to introduce filter function here?}
\begin{align*}
  \atrain = \bigcup_{\aproject \in projects} \aextract(\atime^{-2}, \aproject) \\
  \aval = \bigcup_{\aproject \in projects} \aextract(\atime^{-1}, \aproject) \setminus \atrain \\
  \atest = \bigcup_{\aproject \in projects} \aextract(\atime, \aproject) \setminus \atrain \setminus \aval \\
\end{align*}
