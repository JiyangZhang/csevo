\section{Preliminaries}
\label{sec:prelim}

We briefly define symbols and functions that we will use to define
\methodologies and explain our evaluation setup.  We write \atime{} to
denote a specific point in time; \atimep{} is earlier than \atime{}
(i.e., $\atimep < \atime$), and \atimepp{} is earlier than \atimep{}.
Furthermore, we write \aexamples{} to denote a set of \examples for a ML model.  The
content of a single element is task dependent.  For example, for the
comment generation task, an \example can be a pair (method, comment),
such that method includes its name, signature, and body.  We write
\aprojects{} to denote a set of projects used in evaluation.  There
can be a number of ways to obtain this set of projects.  We consider
project selection to be orthogonal to our work.

We define a function $\ashuffle(Set)$ that takes a set (with \examples
of any type) as the input and returns a list of \examples in a random
order.  A function $\asplit(List, Float, Float, Float)$ takes a list as an
input and percentage of \examples for training, validation, and testing
sets and returns three lists. \JiyangComment{A function $\asplit(List,
  Pecent_{train}, Percent_{val}, Percent_{test})$ takes a list
  and percentage of \examples for training, validation and testing
  sets as inputs and returns three sub-lists split according to the percentage.}
Finally, a function $\aextract(Time,
Project)$ extracts all \examples \JiyangComment{from Project} at a specific time point, which is
given as the first argument.
