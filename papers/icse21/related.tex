\section{Related Work}
\label{sec:related}

Tu et al.~\cite{TuETAL18Careful} revealed the data leakage problem
when using issue tracking data in the literature, caused by the
unawareness of the evolution of issue attributes.  We revealed that
similar problem also exists in the naturalness of software when using
code base data caused by the unawareness of the evolution of code, and
we propose practical and generalizable solutions to evaluate ML
techniques in evolution-aware settings to mitigate the threats of data
leakage.

\cite{BiswasAndRajan20Machine}

\subsection{Code Summarization}

%% \begin{table*}
\begin{small}
\begin{center}
\caption{\TCModels}
\begin{tabular}{l | l | l | l}

\toprule

Model & \makecell{Target\\Language} & Inputs & Code \\

\midrule

Iyer et al.~\cite{IyerETAL16Summarizing} CODE-NN &
\Csharp SQL &
code snippets &
\href{https://github.com/sriniiyer/codenn}{[Torch]} \\
\midrule

Liang and Zhu et al.~\cite{LiangAndZhu18Automatic} Tree-LSTM-GRU &
Java &
AST(exclude method names and params) &
\href{https://github.com/liang2024086/code_comment_generation}{[Tensorflow]} \\
\midrule

Wan et al.~\cite{WanETAL18Improving} Code-RL &
Python &
AST+code snippets &
\href{https://github.com/wanyao1992/code_Summarization_public}{[Pytorch]} \\
\midrule

Hu et al.~\cite{HuETAL18Deep} DeepCom &
Java &
SBT+code snippets &
\href{https://github.com/xing-hu/EMSE-DeepCom}{[Tensorflow]} \\
\midrule

Hu et al.~\cite{HuETAL18Summarizing} TL-CodeSum &
Java &
api sequence+code snippets &
\href{https://github.com/xing-hu/TL-CodeSum}{[Tensorflow]} \\
\midrule

LeClair et al.~\cite{LeClairETAL19Neural} ast-attendgru &
Java &
AST+code snippets &
\href{https://s3.us-east-2.amazonaws.com/icse2018/index.html}{[Keras]} \\
\midrule

Alon et al.~\cite{AlonETAL19code2seq} code2seq &
Java &
AST paths &
\href{https://github.com/m3yrin/code2seq}{[Pytorch]} \\
\midrule

Fernandes et al.~\cite{FernandesETAL19Structured} BiLSTM+GNN &
Java &
graph representation of code &
\href{https://github.com/CoderPat/structured-neural-summarization}{[Tensorflow]}\\
\midrule

LeClair et al.~\cite{LeClairETAL20Improved} code+gnn+BiLSTM &
Java &
AST+code snippets &
\href{https://github.com/acleclair/ICPC2020_GNN}{[Tensorflow+Keras]} \\
\midrule

\bottomrule

\end{tabular}
\end{center}
\end{small}
\vspace{\TVModels}
\end{table*}

Most of the neural comment generation work
frame the task as a encoder-decoder structure
and generate comments from source code. The mainstream
approaches can be categorized into textual-based approach,
structure-based approach and hybrid approach.

The textual-based approaches only consider the sequential
information in the code sequence.
Iyer et al.\cite{IyerETAL16Summarizing} use Long Short Term Memory
(LSTM) network with attention to produce summaries with given SQL
sequence. Allamanis et al.~\cite{AllamanisETAL16Convolutional} propose
to use convolutional attention network to summarize the input source
code. Ahmad et al.~\cite{AhmadETAL20Transformer-based} propose to use transformer equipped with relative positions representation and copy attention to generate comments.

The structure-based approaches mostly takes the structure of source code
into consideration.
Hu et al.~\cite{HuETAL18Deep} develop a novel traversal
method ``Structure-Based Traversal'' to first flatten the Abstract
Syntax Tree (AST) into sequence as the input of encoder.
Liang et al.~\cite{LiangAndZhu18Automatic} use Recursive Neural
Network to directly encode the AST of code snippets. Alon et
al.~\cite{AlonETAL19code2seq} extract AST paths
(path between two terminals in AST) from code and uniformly sample to
get the inputs of LSTM encoder. Fernandes et
al.~\cite{FernandesETAL19Structured} first represent code with graph
following the way proposed by Allamanis et
al.~\cite{AllamanisETAL18Learning}
and use both Gated Graph Neural Network (GGNN) and biLSTM to learn the graph representation.

The hybrid approaches usually use more than one modality to help
generate comments. Hu et al.~\cite{HuETAL19Deep} propose a model to
encode SBT sequence and code sequence separately with separate
attention which shows improvement over the AST based
approach~\cite{HuETAL18Summarizing}. LeClair et
al.~\cite{LeClairETAL20Improved} use a Graph Neural Network to learn
the AST representation rather than directly using flattened AST
sequence.

In addition to AST, other auxiliary information is used to help
comment generation. Hu et al.~\cite{HuETAL18Summarizing} extract and
encode API usage sequence from the source code to see further
improvement. Cai et al.~\cite{CaiETAL20TAG} incorporate the type of
AST node. They first build a Token-type-tree which is a
N-ary tree where each node is consisted of token from code sequence and the type of the corresponding AST node
from source code. Then they use tree-LSTM  encoder and decoder to
generate comments with the help of type information.

% (ICPC'18); Hu et al.~\cite{HuETAL19Deep}
% Based on \sts models, where encoder can take tree inputs by
%flattening them into sequences.  They essentially like this: ``(
%parent children ) parent''.  Handle OOV by backing-up to the AST
%type. Dataset is available.


%Iyer et al.~\cite{IyerETAL16Summarizing} use 
%(ACL'16). CODE-NN, used
%Long Short Term Memory (LSTM) networks with attention to produce summaries
%of \Csharp and SQL based on the code snippet.
%Dataset is from Stack Overflow.

%Loyola et al.~\cite{LoyolaETAL17Neural} (ACL'17). Used seq2seq model
%trained on code changes to generate commit messages.  Dataset is 12
%real-world repositories. Language: python, javascript, c++, java.


%Liang and Zhu~\cite{LiangAndZhu18Automatic}% (AAAI'18)
%take advantage of the structure information of programming language
%and use Recursive NN (tree-lstm) to encode
%the AST of code snippets.
%They added 'CombineName' node which indicates sub-tokenizing
%the identifiers. Then used the learned representation of ast as the input of GRU decoder.
%They studied Java, and used ten open-source Java code repos from Github. 
%Used ten open-source Java code repositories from
%GitHub for this experiment. In each of these
%repositories they extracted descriptive comment and the corresponding method pairs.
%Constructor methods are excluded from this exercise. These pairs are then used for
%training and test. \textbf{Notice that all the method names and
%parameters are excluded from training and test.}
%\textbf{Code}: [Tensorflow] \url{https://github.com/liang2024086/code_comment_generation}

%Reinforcement learning is also applied to this area, Wan et
%al.~\cite{WanETAL18Improving} %(ASE'18).
%incorporate AST as well as sequential content of code snippets into a deep reinforcement learning framework
%(i.e., actor-critic network). The actor network provides the confidence
%of predicting the next word according to current state. The critic network
%evaluates the reward value of all possible extensions of the current state and
%can provide global guidance for explorations.
%They employ an advantage reward composed of BLEU metric to train both networks.

%Studied python, used datasets from~\cite{BaroneAndSennrich17Parallel}
%The experiments are done with Python.
%To construct the tree-structure of code, they parse Python code into
%abstract syntax trees via ast2 lib. To convert code into sequential
%text, they tokenize the code by some special tokens.  They also tokenize the comment by {(space)}.
%``{. , ” ’ : ; ) ( ! (space)}. 
%\textbf{Code}: [Pytorch] \url{https://github.com/wanyao1992/code_Summarization_public}

%% Hu et al.~\cite{HuETAL18Deep}% (ICPC'18); Hu et al.~\cite{HuETAL19Deep}
%% (ESE'19).  Based on \sts models, where encoder can take tree inputs by
%% flattening them into sequences.  They developed a novel flattening
%% method ``Structure-Based Traversal'', essentially like this: ``(
%% parent children ) parent''.  Handle OOV by backing-up to the AST
%% type. Dataset is available.
%To decrease noise introduced to the
%learning process, they only take the first sentence of the comments%
%since they typically describe the functionalities of Java methods
%according to Javadoc guidance2 . However, not every comment is useful.
%Methods with empty or just one-word descriptions are filtered out in
%this work. The setter, getter, constructor, test methods, and override
%methods, whose comments are easy to predict, are also excluded.
%\textbf{Code}: [Tensorflow]
%\url{https://github.com/xing-hu/EMSE-DeepCom}

%% The BLEU scores reported in Hu et al.~\cite{HuETAL18Deep} could not be
%% replicated as pointed out by Alon et al.~\cite{AlonETAL19code2seq}.
%% This might be due to a different nltk library version they used, as we
%% observed the BLEU score computation in nltk library has been
%% dramatically changed in the history (e.g., version 3.2).



%% Hu et al.~\cite{HuETAL18Summarizing} (IJCAI'18). Use a code encoder and API encoder (pre-trained on
%% API summarization task) as inputs of decoder. (API sequence can be obtained from source code,
%% but the code used to extract not provided) 
%% Dataset is available.
%\textbf{Code}: [Tensorflow] \url{https://github.com/xing-hu/TL-CodeSum}
%Methods with empty or just one-word descriptions are filtered
%out in this work. The setter, getter, constructor, test methods,
%and override methods, whose comments are easy to predict,
%are also excluded.

%% LeClair et al.~\cite{LeClairETAL19Neural} (ICSE'19).  In addition to
%% the ``standard'' experiment of summarization from source code, they
%% also study the ``challenge'' experiment of summarization from bytecode
%% (using only AST of methods to generate comments). They create a multi-input model
%% that used the SBT sequence with all identifiers removed as the first
%% input, and the source code tokens as the second. They found that if
%% you decouple the structure of the code form the code itself that the
%% model improved its ability to learn that structure.
%% Dataset from
%% Sourcerer repository provided by Lopes et al. The repository contains
%% over 51 million Java methods from over 50000 projects.
%% They take out first sentence of Javadoc, filter out comments not in English,
%% using langdetect (\url{https://pypi.org/project/langdetect/}) remove
%% methods over 100 words long and comments >13 and <3 words they remove
%% any methods from files that include phrases such as ``generated by''.
%% This filter is quite aggressive, as it reduced the dataset size
%% to around 2m methods, and on manual inspection they found no cases of
%% autogenerated code. In fact, a majority of the filtered methods theyre
%% exact duplicates (around 100k unique examples out of \~2m removed
%% methods). But because comments to auto-generated code are often still
%% meaningful, they added one copy of each of the 100k unique examples back
%% into the dataset, and ensured that they theyre in the training set only
%% (so they did not attempt to test against auto-generated comments). The
%% result is a dataset of around 2.1m methods.  To obtain the ASTs, they
%% first used srcml (\url{https://ieeexplore.ieee.org/document/6065176})
%% to extract an XML representation of each method. Then they built a tool
%% to convert the XML representation into the flattened SBT
%% representation. they created our own modification of SBT in which all
%% the code structure remained intact, but in which they replaced all words
%% (except official Java API class names) in the code to a special
%% <OTHER> token. They call this SBT-AO for SBT AST only. They use this
%% modification to simulate the case when only an AST can be extracted.
%% \textbf{Code}: [Keres]
%% \url{https://s3.us-east-2.amazonaws.com/icse2018/index.html}

%% %LeClair and McMillan~\cite{LeClairAndMcMillan19Recommendations} (NAACL'19).
%% %This paper recommends people to use their dataset to do code summarization.
%% %The dataset includes 2.1m pairs of
%% %Java methods and \javadoc pairs over 28k Java projects.  Dataset based
%% %on their previous work~\cite{LeClairETAL19Neural}, rooted
%% %from~\cite{LopesETAL10UCI}.  Analyzed splitting by method
%% %v.s. splitting by project, and removing auto generated Java methods.
%% %Dataset can be downloaded: \url{http://leclair.tech/data/funcom/}

%% Alon et al.~\cite{AlonETAL19code2seq} (ICLR'19). They combine the uniformly sampled k
%% AST paths (path between two terminals in AST) and terminals' tokens as
%% inputs of decoder, then generate comments. Studied Java, use their own
%% datasets.  Java-large: new dataset of the 9500 top-starred Java
%% projects from GitHub that were created since January 2007. They randomly
%% select 9000 projects for training, 250 for validation and 300 for
%% testing. This dataset contains about 16M examples and it is publicly available.  \textbf{Code}: [Tensorflow]
%% \url{https://github.com/tech-srl/code2seq} [Pytorch]
%% \url{https://github.com/m3yrin/code2seq}

%% Fernandes et al.~\cite{FernandesETAL19Structured} (ICLR'19).  Experimented with GGNN
%% encoder and sequence+GGNN encoder.  Studied both \Csharp and Java,
%% used many datasets from various sources. Represent code with graph~\cite{AllamanisETAL18Learning}.
%% Dataset
%% from~\cite{AlonETAL19code2seq}. \textbf{Code}: [Tensorflow]
%% \url{https://github.com/CoderPat/structured-neural-summarization}

%% LeClair et al.~\cite{LeClairETAL20Improved} (ICPC'20).  Models based
%% on their previous work~\cite{LeClairETAL19Neural}.  Use a ConvGNN
%% rather than flattened AST in the multi-input \sts model.  Dataset
%% taken from their previous
%% work~\cite{LeClairAndMcMillan19Recommendations}.  Use the
%% SrcML(\url{https://www.srcml.org/}) library to generate the associated
%% ASTs from the raw source code.  \textbf{Code}: [Tensorflow+Keras]
%% \url{https://github.com/acleclair/ICPC2020_GNN}

%% Ahmad et al.~\cite{AhmadETAL20Transformer-based} (ACL'20).  They use a
%% transformer equipped with relative encoding of code tokens' positions
%% and copy mechanism.  They use a Java dataset (from Hu et
%% al.~\cite{HuETAL18Summarizing}) and Python dataset (from Barone and
%% Sennrich~\cite{BaroneAndSennrich17Parallel}).  \textbf{Code}:
%% [Pytorch] \url{https://github.com/wasiahmad/NeuralCodeSum}

%% Cai et al.~\cite{CaiETAL20TAG} (ACL'20).  \Fix{TODO}.

%% Zhao et al.~\cite{ZhaoETAL20Survey} is a survey of some (very old)
%% comment generation models.

\subsection{Evolution Data}
Tan et al.~\cite{TanETAL15Online} point out that using
cross-validation to train defect prediction classifier (predict
whether a change is buggy at the time of the commit) will use future
data which does not match a real-world usage scenario. Therefore they
propose time sensitive change classification which uses information,
known at time $t$ only, to classify change $c$ that is committed at $t$.
For example, time sensitive change classification predicts at time
$t_{predict}$ for the change $C_6$, i.e., the test set.  The changes
committed before $C_6$ are the training set, i.e., $C_1$-$C_5$, which
is used to build models.

Lutellier et al.~\cite{LutellierETAL20CoCoNut} propose a new technique
CoCoNuT, which uses ensemble learning on the combination of
convolutional neural networks (CNNs) and a new context-aware neural
machine translation (NMT) architecture to automatically fix bugs in
multiple programming languages. They claim they are the first to
acknowledge and address using-future-data threat in the context of
program repair techniques (APR).
They split benchmark into two parts The first part contains bugs from
2006 to 2010 and is used to evaluate CoCoNuT trained with data from
before 2006. The second part of the benchmark contains bugs from 2011
to 2016 and is used to evaluate CoCoNuT trained with data from before
2011 (including data from before 2006). This split allows CoCoNuT to
learn from instances up to 2010 to fix newer bugs while keeping the
overhead reasonable. They then combine the results of CoCoNuT on these
two sub-benchmarks to obtain the final number of bugs fixed. With this
correct setting, CoCoNuT has no access to data that would be
unavailable in a realistic scenario.

Pradel et al.~\cite{PradelETAL20Scaffle} present a bug localization
technique which is based on the key insight to divide the problem into
two easier sub-problems. First, a trained machine learning model
encodes the raw crash trace and predicts which lines of a raw crash
trace are most informative for localizing the bug. Then, these lines
are fed to an information retrieval-based search engine to retrieve
file paths in the code base, predicting which file to change to
address the crash.  To evaluate the system in a realistic setup, they
simulate using the approach at different points in time. At each point
in time $t$, they simulate by training its model based on all data
available at $t$ and by predicting the bug locations for all crash
traces that occur between $t$ and $t$ + 50 days.  For the prediction,
they gather the set of all files in the code base at $t$ + 50 days and
let the path prediction component predict which of these files need to
be fixed. This setup is realistic, as it uses only past data to
predict future bug locations
